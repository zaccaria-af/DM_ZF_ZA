	\section{Syntax und Semantik am Beispiel der formalen Aussagenlogik}
	\subsection{Syntax der Aussagenlogik}
	\begin{definition}{Definition 5}
		Die \textit{Sprache der Aussagenlogik} (auch Zeichenvorrat genannt) besteht aus:
		\begin{itemize}
			\item Konstanten $\top$ und $\bot$.
			\item Variablen $p,q,r,s,...,p_0,p_1,p_2,...$
			\item Klammern $(,)$
			\item Junktoren $\neg,\land,\lor,\to$
		\end{itemize}
		Die Menge der Variablen bezeichnen wir mit $\mathbb{V}$.
	\end{definition}
	
	\begin{definition}{Definition 6}
		Jede Variable und jede Konstante ist eine \textit{atomare Formel}. Wir bezeichnen die Menge aller
		atomaren Formeln mit $\mathbb{A} \coloneqq \left\{\top,\bot,p,q,r,s,...,p_0,p_1,p_2,...\right\}$. Die
		\textit{Formeln} der Aussagenlogik sind dann wie folgt gegeben:
		\begin{itemize}
			\item Alle atomaren Formeln sind Formeln.
			\item Sind $P$ und $Q$ schon Formeln, dann auch: $(P\land Q),(P\lor Q),(P\to Q), \neg P$
		\end{itemize}
		Wir schreiben $\mathbb{F}$ für die Menge aller aussagenlogischen Formeln.	
	\end{definition}

	\subsection{Semantik der Aussagenlogik}
	\begin{definition}{Definition 7}
		Eine \textit{Belegung} ist eine Zuordnung von Variablen zu Wahrheitswerten, d.h. eine Funktion \\
		$B:\mathbb{V}\to \left\{\texttt{true, false}\right\}$.
	\end{definition}
	
	\begin{definition}{Definition 8}
		Es sei eine Belegung $B$ gegeben. Die Funktion $\hat{B}$ ist die Funktion, die jeder aussagenlogischen
		Formel ihren Wahrheitswert bezüglich der Belegung $B$ zuordnet, d.h. die Funktion
		$\hat{B}:\mathbb{F}\to\left\{\texttt{false,true}\right\}$ ist gegeben durch:
		\begin{itemize}
			\item $\hat{B}(\bot)=\texttt{false}$ und $\hat{B}(\top)=\texttt{true}.$
			\item Für beliebige atomare Formeln $x$ gilt $\hat{B}(x)=B(x)$.
			\item Für beliebige Formeln $F$ und $G$ gilt
			\begin{equation*}
				\hat{B}(F\land G) = 
				\begin{cases}
					\texttt{true} &\text{falls } \hat{B}(F)=\texttt{true } \text{und } \hat{B}(G)=\texttt{true} \\
					\texttt{false} &\text{sonst}
				\end{cases}
			\end{equation*}
			\item Für beliebige Formeln $F$ und $G$ gilt
			\begin{equation*}
				\hat{B}(F\lor G) = 
				\begin{cases}
					\texttt{true} &\text{falls } \hat{B}(F)=\texttt{true } \text{oder } \hat{B}(G)=\texttt{true} \\
					\texttt{false} &\text{sonst}
				\end{cases}
			\end{equation*}
			\item Für beliebige Formeln $F$ gilt
			\begin{equation*}
				\hat{B}(\neg F) = 
				\begin{cases}
					\texttt{true} &\text{falls } \hat{B}(F)=\texttt{false} \\
					\texttt{false} &\text{sonst}
				\end{cases}
			\end{equation*}
			\item Für beliebige Formeln $F$ und $G$ gilt $\hat{B}(F\to G)=\hat{B}(\neg F \lor G)$
		\end{itemize}
	\end{definition}

\subsection{Wahrheitstabellen}
	\begin{definition}{Definition 9}
		Der Begriff einer Teilformel einer Formel $F$ ist wie folgt gegeben: \\
		\begin{itemize}
			\item Wenn $F$ eine atomare Formel ist, dann besitzt $F$ nur die Teilformel $F$.
			\item Wenn $F$ von der Form $A \lor B$, $A \land B$ oder $A \rightarrow B$ ist,
				dann besitzt $F$ als Teilformeln, neben $F$ selbst, alle Teilformeln von
				$A$ und $B$.
			\item Wen $F$ von der Form $\neg A$ ist, dann besitzt $F$ als Teilformeln, neben
				$F$ selbst, alle Teilformeln von $A$.
		\end{itemize}
	\end{definition}
	\begin{definition}{Definition 10}
		In einer \textit{Wahrheitstabelle} einer Formel $F$ entspricht jede Spalte einer Teilformel
		von $F$ und jede Zeile einer Belegung der in $F$ vorkommenden Variablen. Es gelten folgende
		Eigenschaften:
		\begin{itemize}
			\item In der ersten Zeile stehen Formeln un in allen anderen Zeilen stehen
				Wahrheitswerte.
			\item Der letzte Eintrag der ersten Zeile ist die Formel $F$.
			\item Für Formeln $A$ und $B$ gilt: Wenn $A$ in einer Spalte vor (weiter Links) als B
				erscheint, dann ist $A$ eine Teilformel von $B$.
			\item Mit jeder Formel erscheinen auch alle ihre Teilformeln in der Tabelle.
		\end{itemize}
	\end{definition}

\subsection{Semantische Eigenschaften}
\begin{definition}{Definition 11 \& 12}
	Eine aussagenlogische Formel $A$ heisst
	\begin{itemize}
		\item \textit{Gültig} oder \textit{wahr} unter einer Belegung $B$, falls $\hat{B}(A)=$
			\texttt{true}.
		\item \textit{Allgemeingültig}, wenn sie unter jeder Belegung gültig ist.
		\item \textit{Unerfüllbar}, wenn $A$ nicht erfüllbar ist.
		\item \textit{Widerlegbar}, wenn es mindestens eine Belegung gibt, unter der $A$ nicht gültig
			ist.
	\end{itemize}
	Es seien $F$ und $G$ beliebige aussagenlogische Formeln. Wir sagen
	\begin{itemize}
		\item $F$ ist eine \textit{Konsequenz} von $G$, falls $F$ unter jeder Belegung wahr ist unter der $G$ wahr
			ist.
		\item $F$ und $G$ sind \textit{logisch äquivalent}, wenn $G$ $F$ unter jeder Belegung denselben
			Wahrheitswert annehmen.
	\end{itemize}
	Sind $F$ und $G$ äquivalente Formeln, dann schreiben wir $F\equiv G$.
\end{definition}

\textbf{Satz 1} \textit{Sind F, G und H beliebige aussagenlogische Formeln, dann gelten folgende Äquivalenzen:}
\begin{itemize}
	\item \textit{Gesetz der doppelten Negation: } $\neg\neg F \equiv F$
	\item \textit{Absorbtion: } $F \land F  \equiv F$ \textit{ und } $F \lor  F \equiv F$
	\item \textit{Kommutativität: } $F \land G \equiv G \land F$ \textit{ und } $F \lor G \equiv G \lor F$
	\item \textit{Assoziativität: } $F \land (G \land H) \equiv (F \land G) \land H$
	\item \textit{Assoziativität: } $F \lor (G \lor H) \equiv (F \lor G) \lor H$
	\item \textit{Distributivität: } $F \land (G \lor H) \equiv (F \land G) \lor (F \land H)$
	\item \textit{Distributivität: } $F \lor (G \land H) \equiv (F \lor G) \land (F \lor H)$
	\item \textit{De Morgan: } $\neg(F \land G) \equiv \neg F \lor \neg G$
	\item \textit{De Morgan: } $\neg (F \lor G) \equiv \neg F \land \neg G$
	\item \textit{Kontraposition: } $F \rightarrow G \equiv \neg G \rightarrow \neg F$ 
\end{itemize}

\textbf{Theorem 1} \textit{Sind F und G aussagenlogische Formeln, dann gelten:}
\begin{enumerate}
	\item \textit{G ist genau dann eine Konsequenz von F, 
		wenn die Formel $F \rightarrow G$ allgemeingültig ist.}
	\item \textit{F und G sind genau dann logisch äquivalent, 
			wenn die Formel $F \rightarrow G \land G \rightarrow F$ allgemeingültig ist.}
\end{enumerate}

