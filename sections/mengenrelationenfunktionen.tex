\section{Mengen, Relationen und Funktionen}%
\label{sec:mengen_relationen_und_funktionen}
\subsection{Mengen}%
\label{sub:mengen}

\subsubsection{Schreibweise und Mengenbegriff}%
\label{ssub:schreibweise}
\begin{minipage}{0.9\linewidth}
Zwei Mengen sind genau dann gleich, wenn sie dieselben Elemente enthalten:
$X = Y\, \Leftrightarrow \forall z\, (z \in X \Leftrightarrow z \in Y)$. \\
Sind mathematische Objekte $x_1,\dots,x_n$ gegeben, dann schreiben wir
\[
\{x_1,\dots,x_n\}
\]
für die Menge die als Elemente genau $x_1,\dots,x_n$ hat.
$\{\,\}$: \textit{leere Menge} ($\varnothing$), einzige Menge, die keine Elemente besitzt. 
\end{minipage}

\subsubsection{Teilmengen}%
\label{ssub:teilm}
\begin{minipage}{0.9\linewidth}
$X\subseteq Y$ ($X$ ist eine \textit{Teilmenge} von $Y$), wenn jedes Element von $X$ auch ein Element von $Y$ ist:
\[
X\subseteq Y:\,\Leftrightarrow\,\forall x\,(x\in X\Rightarrow x\in Y).
\]
$X\subsetneq Y$ ($X$ ist eine \textit{echte Teilmenge} von $Y$), falls $X$ eine von $Y$ verschiedene Teilmenge von $Y$ ist:
\[
X\subsetneq Y\,:\Leftrightarrow\, X\subseteq Y\land X\neq Y.
\]
\end{minipage}

\subsubsection{Prädikative Schreibweise}%
\label{ssub:mengeneigenschaft}
\begin{minipage}{0.9\linewidth}
Menge aller Elemente $z$ von $X$ mit der Eigenschaft $\mathsf{E}(z)$.	
\[
\big\{z\in X\mid \mathsf{E}(z)\big\}
\]
oder
\[
\big\{z\mid z\in X\land\mathsf{E}(z)\big\}
\]
\end{minipage}

\subsubsection{Ersetzungsschreibweise}%
\label{ssub:ersetzungsschreibweise}
\begin{minipage}{0.9\linewidth}
\[
\big\{F(x)\mid x\in X \big\}
\]
Menge aller Funktionswerte $F(x)$.
\[
\big\{F(x)\mid x\in X\big\}:=\{y\mid \exists x\in X\,(y=F(x))\}.
\]
\end{minipage}

\subsubsection{Schnittmenge und Vereinigung}%
\label{ssub:schnittmenge_und_vereinigung}
\begin{minipage}{0.9\linewidth}
\[
X\cup Y:=\{x\mid x\in X\lor x\in Y \}
\]
\textit{Vereinigung} von $X$ mit $Y$. \textit{Schnittmenge} von $X$ und $Y$:
\[
X\cap Y:=\{x\in X\mid x\in Y \}=\{x\mid x\in X\land x\in Y\}
\]
Ist $I$ eine Menge so, dass für alle Elemente $i\in I$ auch $A_i$ eine Menge ist, dann wird
\[
\bigcup_{i\in I}A_i:=\{x\mid\exists i\in I\,(x\in A_i) \}.
\]
die Vereinigung von $\{A_i\mid i\in I\}$ genannt.
Analog dazu, ist die \textit{Schnittmenge} durch
\[
\bigcap_{i\in I}A_i:=\{x\mid\forall i\in I\,(x\in A_i) \}
\]
gegeben, falls $I\neq\varnothing$ ist.
\end{minipage}

\subsubsection{Disjunkte Mengen}%
\label{ssub:disjunkte_mengen}
\begin{minipage}{0.9\linewidth}
\textit{disjunkt}: zwei Mengen, die keine gemeinsamen Elemente haben, d.h. \\
$X\cap Y=\varnothing$
\textit{paarweise disjunkt}:
 \[
 \forall i,j\in I\,(i\neq j\Rightarrow A_i\cap A_j=\varnothing).
 \]
\end{minipage}

\subsubsection{Komplementärmenge}%
\label{ssub:komplementärmenge}
\begin{minipage}{0.9\linewidth}
 \[
 X\setminus Y:=\{x\in X\mid x\notin Y\}
 \]
Menge aller Elemente von $X$, die nicht zu $Y$ gehören.
\end{minipage}

\subsubsection{Rechenregeln für Mengen}%
\label{ssub:rechenregeln_für_mengen}
\begin{minipage}{0.9\linewidth}
\begin{enumerate}
 \item Kommutativität der Vereinigung und des Schnittes:
\[
 A\cup B=B\cup A\
 A\cap B=B\cap A.
\]
 \item Assoziativgesetze von Schnitt und Vereinigung:
\[
 A\cap(B\cap C)=(A\cap B)\cap C
 A\cup(B\cup C)=(A\cup B)\cup C
\]
 \item Distributivgesetze von $\cap$ mit $\cup$:
\[
 A\cap(B\cup C)=(A\cap B)\cup (A\cap C)
 A\cup(B\cap C)=(A\cup B)\cap (A\cup C)
\]
 \item Idempotenzgesetz:
\[
  A\cap A=A\
  A\cup A=A
\]
 \item Regeln von DeMorgan:
\[
 (C\backslash A)\cap (C\backslash B)=C\backslash (A\cup B)\\
 (C\backslash A)\cup (C\backslash B)=C\backslash (A\cap B)
\]
 \item Charakterisierung der Teilmengenbeziehung:
\[
 A\subseteq B\Leftrightarrow A\cap B= A\Leftrightarrow A\cup B=B
\]
\end{enumerate}
\end{minipage}

\subsubsection{Potenzmenge}%
\label{ssub:potenzmenge}
\begin{minipage}{0.9\linewidth}
\[
 \mathcal{P}(A):=\{x\mid x\subseteq A\}
\]
\textit{Potenzmenge} von $A$, die genau die Teilmengen von $A$ als Elemente enthält.
\end{minipage}

\subsubsection{Partition}%
\label{ssub:partition}
\begin{minipage}{0.9\linewidth}
\textit{Partition} $P=\{P_i\mid i\in I \}$ = Menge von Teilmengen von $A$, die folgende beiden Voraussetzungen erfüllt:
\begin{itemize}
 \item Die Elemente von $P$ sind nichtleer und paarweise disjunkt.
 \item $\bigcup_{i\in I}P_i=A$
\end{itemize}
Elemente einer Partition = \textit{Blöcke}.
\end{minipage}

\subsection{Relationen, Funktionen und Graphen}%
\label{sub:Relatiinen, Funktionen und Graphen}

\subsubsection{Tupel}%
\label{ssub:tupel}
Ein \textit{n-Tupel} ist ein Term der Form $(x_1,...,x_n)$ wobei $n$ eine natürliche Zahl grösser $0$ ist. \\
Für beliebige Tupel gilt:
\begin{equation}
	(x_1,\dots,x_n)=(y_1,\dots,y_k):\Leftrightarrow n=k\land x_1=y_1\land\dots\land y_n=x_n.
\end{equation}
2-Tupel nennen wir \textit{Paare} und 3-Tupel \textit{Tripel}.

\subsubsection{Kartesisches Produkt}%
\label{ssub:kartesisches_produkt}
\begin{minipage}{0.9\linewidth}
\begin{align*}
\prod_{i=1}^{n}A_i=\big\{(a_1,\dots,a_n)\mid a_1\in A_1\land\dots\land a_n\in A_n \big\}.
\end{align*}
Das \textit{kartesische} Produkt von $A_1, ..., A_n$ ist die Menge aller $n$-Tupel mit Einträgen aus den Mengen $A_1,
..., A_n$ 
\\
Für das kartesische Produkt von zwei Mengen schreibt man:
\[
X\times Y:=\{(x,y)\mid x\in X\land y\in Y \}.
\]
\end{minipage}

\subsubsection{Relationen}%
\label{ssub:relationen}
\begin{minipage}{0.9\linewidth}
Eine $n$-Stellige \textit{Relation R} ist eine Menge von $n$-Tupeln aus $A_1 \times \dots \times A_n$.
\begin{align*}
R\subseteq A_1\times\dots \times A_n.
\end{align*}
Ist $R$ eine $n$-stellige Relation und gilt $(x_1,\dots,x_n)\in R$, dann sagen wir, dass die Elemente $x_1,\dots,x_n$ zueinander in Relation $R$ stehen.
Eine $2$-stellige Relation $R\subseteq X\times Y$ heisst auch eine \textit{binäre Relation} auf den Mengen $X$ und $Y$. Ist $R$ eine binäre Relation, so schreiben wir auch $xRy$ für $(x,y)\in R$.
\end{minipage}


