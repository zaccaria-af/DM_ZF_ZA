\section{Elementare Zahlentheorie}
\subsection{Rechenregeln auf $\mathbb{Z}$}

\begin{align*}
	-1\cdot z & =-z                                                                 \\
	-(-z)     & =z                                                                  \\
	-z+z      & =0                      & \text{ Inverse Elemente bezüglich }+      \\
	0\cdot z  & =0                      & \text{ Absorbtion}                        \\
	1\cdot z  & =z                      & \text{ Neutrales Element bezüglich }\cdot \\
	0+z       & =z                      & \text{ Neutrales Element bezüglich }+     \\
	r(sz)     & =(rs)z                  & \text{ Assoziativität von } \cdot         \\
	r+(s+z)   & =(r+s)+z                & \text{ Assoziativität von }+              \\
	rs        & =sr                     & \text{ Kommutativität von }\cdot          \\
	r+s       & =s+r                    & \text{ Kommutativität von }+              \\
	r(s+z)    & =rs+sz                  & \text{ Distributivität}                   \\
	rx=ry     & \Rightarrow x=y\lor r=0 & \text{Kürzbarkeit}
\end{align*}

\subsubsection{Subtraktion}
Wir definieren die \textit{Subtraktion}
\[
	-:\mathbb{Z}\times\mathbb{Z}\rightarrow\mathbb{Z}
\]
durch
\[
	x-y:=x+(-y),
\]
die \textit{Betragsfunktion}
\[
	|\cdot|:\mathbb{Z}\rightarrow\mathbb{N}
\]
durch
\[
	|z|=\begin{cases}
		z         & \text{falls } z\in\mathbb{N} \\
		-1\cdot z & \text{sonst}
	\end{cases}
\]
und die Relation $\leq$ durch
\[
	x\leq y:\Leftrightarrow\,\exists n\in\mathbb{N}\,(x+n=y).
\]

\subsection{Teilbarkeit und euklidischer Algorithmus}
Sind $x,y\in\mathbb{Z}$ ganze Zahlen, so sagen wir, dass $x$ \textit{ein Teiler von} $y$ ist, falls es ein $k\in\mathbb{Z}$ gibt mit $xk=y$. Wir schreiben in diesem Fall $x|y$. Es gilt also
\[
	x|y:\Leftrightarrow \exists k\in\mathbb{Z}(y=xk).
\]
Mit $T(y)$ bezeichnen wir die Menge aller natürlichen Zahlen, welche Teiler von $y$ sind, also $T(y)=\{x\in\mathbb{N}\mid x|y\}$.

Die Teilbarkeitsrelation ist reflexiv und transitiv auf der Menge $\mathbb{Z}$

\subsubsection{Teilen mit Rest}
Sind $n,m\in\mathbb{N}\backslash\{0\}$, dann gibt es eindeutig bestimmte Zahlen $k,r\in\mathbb{N}$, so dass Folgendes gilt:
\begin{enumerate}
	\item $m=kn+r$
	\item $r<n$
\end{enumerate}
Wir sagen in diesem Zusammenhang, dass die Zahl $r$ den \textit{Rest} von der (ganzzahligen) Division von $m$ durch $n$ ist.

\subsubsection{kgV und ggT}
Seien $n,m\in\mathbb{Z}$. Wir definieren das \textit{kleinste gemeinsame Vielfache von $n$ und $m$} als
\[
	kgV(n,m):=\min\{k\in\mathbb{N}\mid n|k\wedge m|k\}.
\]
Ist $n\neq0$ oder $ m\neq 0$, dann definieren wir den \textit{grössten gemeinsamen Teiler} von $n$ und $m$ als
\[
	ggT(n,m):=\max\{k\in\mathbb{N}\mid k|n\wedge k|m\}.
\]

Sind $x,y,z\in\mathbb{Z}$, dann sind folgende Aussagen äquivalent:
\begin{enumerate}
	\item[1.] $ x|y\wedge x|z$
	\item[2.] $x|y\wedge x|(y-z) $
\end{enumerate}

$1.\Rightarrow 2.$: Wenn $x|y\wedge x|z$, dann gibt es ganze Zahlen $k,k'\in\mathbb{Z}$, so dass $y=kx$ und $z=k'x$. Es gilt also $y-z=kx-k'x=(k-k')x$.

$2.\Rightarrow 1.$: Es seien $k,k'\in\mathbb{Z}$, so dass $y=kx$ und $y-z=k'x$. Durch Einsetzen erhält man $ kx-z=k'x $ und somit $z=kx-k'x=x(k-k')$.

\subsubsection{Euklidischer Algorithmus}
Für $n,m\in\mathbb{N}$ mit $0<n< m$ gilt
\[
	ggT(n,m)=ggT(n,m-n)=ggT(m,m-n).
\]
Es folgt für $n,m\in\mathbb{N}$ mit $n<m$
\[
	\{k\in\mathbb{N}\mid k|n\wedge k|m\}=\{k\in\mathbb{N}\mid k|n\wedge k|(m-n)\}.
\]
Daraus folgt weiter
\[
	ggT(n,m)=\max\{k\in\mathbb{N}\mid k|n\wedge k|m\}=
\]
\[
	\max\{k\in\mathbb{N}\mid k|n\wedge k|(m-n)\}=ggT(n,m-n).
\]
Die Gleichung
\[
	ggT(n,m)=ggT(m,m-n)
\]
folgt analog aus Lemma 4.
\\
Aus dem eben bewiesenen Satz erhalten wir direkt einen rekursiven Algorithmus zur Berechnung des $ggT$. Beispielhaft geht man dabei wie folgt vor:
\begin{align*}
	ggT(45,25) & \stackrel{Satz~\ref{satz:euklid}}{=}ggT(25,20)  \\
	           & \stackrel{Satz~\ref{satz:euklid}}{=}ggT(20,5)   \\
	           & \stackrel{Satz~\ref{satz:euklid}}{=}ggT(5,15)   \\
	           & \stackrel{Satz~\ref{satz:euklid}}{=}ggT(5,10)   \\
	           & \stackrel{Satz~\ref{satz:euklid}}{=}ggT(5,5)=5.
\end{align*}


Betrachten wir nochmals den Satz~\ref{satz:euklid}, dann sehen wir, dass wir mehrere Schritte zu einem einzigen Schritt zusammenfassen können. Bei $x>y$ wird nämlich, zum Berechnen von $ggT(y,x)$ so oft $y$ von $x$ subtrahiert, bis das Resultat kleiner oder gleich $y$ ist. Man kann all diese Subtraktionen also durch eine einzige Division mit Rest ersetzen.
Die beispielhafte Berechnung von $ggT(45,25)$ können wir nun als $2$ Divisionen mit Rest darstellen:
\begin{align*}
	45 & = 1 \cdot 25 + 20                           \\
	25 & = 1 \cdot 20 + \underbrace{5}_{ggT(45,25)}.
\end{align*}
Zusammenfassend stellen wir fest, dass
\[
	ggT(y,x)=ggT(y,R(x,y))
\]
mit
\[
	R(x,y)=\text{ der Rest der Division von }x\text{ durch }y
\]
gilt. Die Funktion $R(x,y)$ steht in vielen Programmiersprachen als ``modulo Funktion'' zur Verfügung und wird im Quellcode oft durch das Prozentzeichen $\%$ aufgerufen. Dies eröffnet die Möglichkeit den euklidischen Algorithmus kompakter zu notieren:\\

\subsubsection{Lemma von Bézout}

Sind $x,y\in\mathbb{Z}$ mit $x,y\neq 0$, dann gibt es ganze Zahlen $a,b$ so dass
\[
	ggT(x,y)=ax+by
\]
gilt.

Wir beweisen das Theorem exemplarisch für den Fall, dass $ggT(m,n)=1$ gilt. Ohne Einschränkung sei $m>n$. Sind $x,y$ beliebige ganze Zahlen, dann bezeichnen wir
\begin{align*}
	R(x,y):=\begin{cases}
		\text{ Der Rest von der ganzzahligen Division von $x$ durch $y$} & \text{falls }x,y>0 \\
		0                                                                & \text{sonst.}
	\end{cases}
\end{align*}
Wir definieren rekursiv eine absteigende Folge $(r_i)_{i\in\mathbb{N}}$ von natürlichen Zahlen wie folgt:
\begin{align*}
	r_i=\begin{cases}
		m                  & \text{falls }i=0 \\
		n                  & \text{falls }i=1 \\
		R(r_{i-2},r_{i-1}) & \text{sonst.}
	\end{cases}
\end{align*}
Da es keine echt absteigende Folge von natürlichen Zahlen gibt, muss die Folge der $(r_i)_{i\in\mathbb{N}}$ stationär werden. Es folgt also aus der Definition der Folge $(r_i)_{i\in\mathbb{N}}$, dass es ein $p\in\mathbb{N}$ gibt, so dass $r_{p}\neq0$ und für alle $p`>p$ gilt $r_{p`}=0$. Es sei $(\lambda_i)_{i\in\mathbb{N}}$ die durch $(r_i)_{i\in\mathbb{N}}$ eindeutig bestimmte Folge natürlicher Zahlen mit der Eigenschaft (*):
\begin{align*}
	r_0     & =\lambda_0\cdot r_1+r_2           \\
	r_1     & =\lambda_1\cdot r_2+r_3           \\
	        & \vdots                            \\
	r_{p-2} & =\lambda_{p-2}\cdot r_{p-1}+r_{p}
\end{align*}
\textbf{Behauptung}: $r_p=1$\\
\textbf{Beweis}: Wir zeigen, dass $r_p$ ein Teiler von allen $r_i$ mit $i\leq p$ ist. Weil $r_0=m,r_1=n$ teilerfremd sind, gilt dann $r_p=1$. Wir beweisen mit (der allgemeinen Version von) Induktion für alle $k\in\mathbb{N}$, dass entweder $r_p|r_{p-k}$ oder $k>p$ gilt.
Falls $k>p$ ist, dann sind wir fertig. Wir können also ohne Einschränkung der Allgemeinheit annehmen, dass $k\leq p$ gilt. Nach Induktionsannahme ist nun $r_p$ ein Teiler von $r_{p-(k-1)}$ und von $r_{p-(k-2)}$, es gibt also ganze Zahlen $x,y$ mit $x\cdot r_p=r_{p-(k-1)}$ und $y\cdot r_p=r_{p-(k-2)}$. Insgesamt haben wir dann
\begin{align*}
	r_{p-k}=\lambda_{p-k}\cdot r_{p-k+1}+r_{p-k+2}=\lambda xr_{p}+yr_{p}=r_{p}(\lambda x+y)
\end{align*}
und somit wie gewünscht, dass $r_p$ ein Teiler von $r_{p-k}$ ist.\\
Wir können nun das Gleichungssystem $(*)$ als
\begin{align*}
	r_0     & =\lambda_0\cdot r_1+r_2       \\
	r_1     & =\lambda_1\cdot r_2+r_3       \\
	        & \vdots                        \\
	r_{p-2} & =\lambda_{p-2}\cdot r_{p-1}+1
\end{align*}
schreiben. Dies ist jedoch mit
\begin{align*}
	1                            & =r_{p-2}-\lambda_{p-2}r_{p-1}                                   \\
	r_{p-1}                      & =r_{p-3}-\lambda_{p-3}r_{p-2}                                   \\
	                             & \vdots                                                          \\
	r_{p-i}                      & =r_{p-i-2}-\lambda_{p-i-2}r_{p-i-1}                             \\
	                             & \vdots                                                          \\
	\underbrace{r_{p-p+2}}_{r_2} & =\underbrace{r_{p-p}}_{m}-\lambda_{0}\underbrace{r_{p-p+1}}_{n}
\end{align*}
äquivalent. Indem wir nun sukzessiv (von unten beginnend) in jeder Zeile des Gleichungssystems die $r_i$ auf der rechten Seite durch eine Summe von Vielfachen von $n$ und $m$ ersetzen, erhalten wir zuoberst im Gleichungssystem für geeignete $s,\delta_i,\gamma_i$ eine Gleichung von der gewünschten Gestalt
\[
	1=\sum_{i=1}^{s}\delta_in-\gamma_im=\sum_{i=1}^{s}\delta_in-\sum_{i=1}^{s}\gamma_im=n\sum_{i=1}^{s}\delta_i-m\sum_{i=1}^{s}\gamma_i.\qedhere
\]

\subsection{Primzahlen}

Eine natürliche Zahl $p \in \mathbb{N}$ ist eine \textit{Primzahl}, wenn $|T(p)|=2$ gilt. Die Menge aller Primzahlen bezeichnen wir mit $\mathbb{P}$. \\
Ist $p$ eine Primzahl, dann gilt $T(p)=$ \\

